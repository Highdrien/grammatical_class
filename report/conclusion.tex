\section{Conclusion}

**Conclusion**

En conclusion, Nous avons exploré deux tâches principales : l'étiquetage des parties du discours (\textit{pos}) et la prédiction des traits morphologiques (\textit{morphy}). 

Le modèle \textit{GET\_POS} a démontré des performances impressionnantes dans la prédiction des parties du discours, avec une accuracy micro de 94.4\% et une accuracy macro de 81.6\% sur le jeu de test. Ces résultats témoignent de la capacité du modèle à apprendre efficacement la structure grammaticale des phrases en français.

Pour la tâche de prédiction des traits morphologiques, trois architectures de modèles ont été développées : \textit{SUPERTAG}, \textit{SEPARATE}, et \textit{FUSION}. Chacun de ces modèles a présenté des performances variées. Le modèle \textit{SEPARATE} a obtenu une accuracy micro de 89.3\% et une métrique \textit{all good} de 4.6\%, montrant sa capacité à prédire les traits morphologiques pour chaque classe. Cependant, le modèle \textit{FUSION} a montré une instabilité lors de l'entraînement, ce qui peut nécessiter des ajustements pour améliorer ses performances.

Les résultats obtenus ouvrent des perspectives intéressantes pour l'amélioration des modèles de langage, en particulier dans le contexte du français. Des ajustements futurs pourraient inclure des stratégies plus sophistiquées pour la gestion des mots inconnus et des expériences avec d'autres architectures de modèles.

En résumé, ce projet a permis d'explorer différentes facettes de la modélisation du langage, de la prédiction des parties du discours à la prédiction des traits morphologiques. Les résultats obtenus constituent une base solide pour des développements futurs dans le domaine de la compréhension automatique du langage naturel en français.

